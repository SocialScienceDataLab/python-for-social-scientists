% !TeX program = lualatex
\documentclass[10pt, aspectratio=169]{beamer}
\usepackage{pgfpages}
\usepackage{fontawesome}
\usepackage[english]{babel}
\usepackage[utf8]{inputenc}
\usepackage{graphics}
\usepackage[T1]{fontenc}
\usepackage[]{csquotes}

% Bibliography (choose backend according to your preferences)
\usepackage[authordate, natbib, useprefix=true, isbn=false,url=false, doi=false, backend=biber]{biblatex-chicago}  
\bibliography{library.bib}
\setbeamertemplate{bibliography item}[text]

% Size of bibliography entries
\renewcommand*{\bibfont}{\footnotesize}

% new citep command
%\usepackage{fixltx2e}
\newcommand{\cpsh}[2][]{\textsuperscript{\fontsize{6}{6}\textcolor{mzescyan}{[{#1}\citealp[]{#2}]}}}
\newcommand{\cpsl}[2][]{\textsubscript{\fontsize{6}{6}\textcolor{mzescyan}{[{#1}\citealp[]{#2}]}}}
\newcommand{\csh}[2][]{\textsuperscript{\fontsize{6}{6}\textcolor{mzescyan}{[{#1}\citealp[]{#2};}}}
\newcommand{\csl}[2]{\textsubscript{\fontsize{6}{6}\textcolor{mzescyan}{\hphantom{[{#2}}\citealp{#1}]}}}
\newcommand{\cpshl}[3][]{\textcolor{mzescyan}{\rlap{\csh[#1]{#2}}{\csl{#3}{#1}}}}
\newcommand{\cpsm}[2][]{\textcolor{mzescyan}{\fontsize{6}{6} [{#1}\citealp[]{#2}]}}

% new color emphasize commands
\newcommand{\cemph}[1]{\textcolor{mzescyan}{#1}}
\newcommand{\gemph}[1]{\textcolor{mzesgold}{#1}}

% Layout
\mode<presentation>
{
	\usetheme[progressbar=frametitle]{metropolis}
	\useoutertheme[]{metropolis}
	\useinnertheme[]{circles}
	
	
	%remove navigation symbols
	\setbeamertemplate{navigation symbols}{}
	
	% define colors
	\definecolor{mzescyan}{RGB}{0,118,150}
	\definecolor{mzesgold}{RGB}{209,163,84}
	\definecolor{mzesdarkgold}{RGB}{127, 110, 87}
	\definecolor{mzesbg}{RGB}{255,255,255} % white
	% \definecolor{mzesbg}{RGB}{252,253,254} % grayish
	
	% Blocks
	\setbeamercolor{block body}{bg=mzescyan!10, fg=black}
	\setbeamercolor{block body alerted}{bg=normal text.bg!90!black}
	\setbeamercolor{block body example}{bg=normal text.bg!90!black}
	\setbeamercolor{block title alerted}{use={normal text,alerted text},fg=mzesgold,bg=black}
	\setbeamercolor{block title}{bg=mzescyan, fg = mzesbg}
	\setbeamercolor{block title example}{use={normal text},fg=example text.fg!75!normal text.fg,bg=normal text.bg!75!black}
	
	% Frames
	\setbeamercolor{fine separation line}{fg=mzesgold}
	\setbeamercolor{item projected}{fg=mzesbg}
	
	% Miniframes
	\setbeamercolor{section in head/foot}{fg=mzescyan,bg=mzesbg}
	
	% Title
	\setbeamercolor{background}{bg=mzesbg}
	\setbeamercolor{background canvas}{bg=mzesbg}
	\setbeamercolor{title}{fg=mzescyan, bg = mzescyan!5}
	\setbeamercolor{titlelike}{fg=mzescyan}
	\setbeamercolor{subsection in head/foot}{bg=mzesgold!10, fg = mzescyan}
	\setbeamercolor{author in head/foot}{bg=mzescyan, fg = mzesbg}
	\setbeamercolor{title in head/foot}{bg=mzescyan!5, fg = mzescyan}
	\setbeamercolor{date in head/foot}{bg=mzesbg, fg = mzescyan}
	\setbeamercolor{page in head/foot}{bg=mzesbg, fg = mzesgold}
	\setbeamercolor{frametitle}{fg=mzescyan, bg = mzesbg}
	\setbeamercolor{item}{fg=mzescyan}
	\setbeamercolor{normal text}{bg=mzesbg,fg=black}
	\setbeamercolor{alerted text}{bg=mzesbg,fg=mzesgold}
	\setbeamercovered{invisible}
}

% Font
\setbeamerfont{frametitle, frametitle continuation}{size=\large}
\setbeamerfont{titlelike}{size=\large}
\setbeamerfont{footline}{size=\scriptsize}

% Logo
\logo{\includegraphics[width=12em]{mzes-logo-solo-4c.eps}}
\newcommand{\nologo}{\setbeamertemplate{logo}{}}
% customize foot line
\makeatother
\setbeamertemplate{footline}
{
	\leavevmode%
	\hbox{%
		\begin{beamercolorbox}[wd=.25\paperwidth,ht=2.25ex,dp=1ex,center]{author in head/foot}%
			\usebeamerfont{author in head/foot}\insertshortauthor
		\end{beamercolorbox}%
		\begin{beamercolorbox}[wd=.6\paperwidth,ht=2.25ex,dp=1ex,center]{title in head/foot}%
			\usebeamerfont{title in head/foot}\insertshorttitle
		\end{beamercolorbox}%
		\begin{beamercolorbox}[wd=.15\paperwidth,ht=2.25ex,dp=1ex,center]{page in head/foot}%
			\insertframenumber{} / \inserttotalframenumber\hspace*{1ex}
	\end{beamercolorbox}}%
	\vskip0pt%
}
\makeatletter

% Change the width of the progress bar to make it more visible (Code taken from here: https://github.com/matze/mtheme/issues/237)
\makeatletter
\setlength{\metropolis@titleseparator@linewidth}{0.5pt} % Title page
\setlength{\metropolis@progressonsectionpage@linewidth}{1pt} % Progress bar on section page
\setlength{\metropolis@progressinheadfoot@linewidth}{1pt} % Progress bar in header
\makeatother

\setbeamertemplate{navigation symbols}{}
\providecommand*\email[1]{\href{mailto:#1}{#1}}

% title slide
\title[Getting Started with Python] % (optional, use only with long paper titles)
{\large \textbf{Getting Started With Python}} 
\subtitle{\small A \textit{How-To} Guide for Social Scientists}

\author[Bach \& K\"{u}pfer]{%
	\texorpdfstring{
		\begin{columns}
			\begin{column}{0.45\textwidth}
				Ruben Bach \\
				\scriptsize University of Mannheim \vspace{1em} \\
				\tiny \hspace*{1em} \faEnvelope \hspace{.5em} \texttt{\email{r.bach@uni-mannheim.de}} \\
%				\tiny \hspace*{1em} \faGlobe \hspace{.5em}  \texttt{\href{https://denis-cohen.github.io/}{denis-cohen.github.io}} \\
				\tiny \hspace*{1em} \faTwitter \hspace{.5em} \texttt{\href{https://twitter.com/rub3n_luc}{@rub3n\_luc}}
			\end{column}
			\begin{column}{0.45\textwidth}
				Andreas K\"{u}pfer \\
				\scriptsize Technical University of Darmstadt \vspace{1em} \\
				\tiny \hspace*{1em} \faEnvelope \hspace{.5em} \texttt{\email{andreas.kuepfer@tu-darmstadt.de}} \\
				\tiny \hspace*{1em} \faGlobe \hspace{.5em}  \texttt{\href{https://andreaskuepfer.github.io/}{andreaskuepfer.github.io}} \\
				\tiny \hspace*{1em} \faTwitter \hspace{.5em} \texttt{\href{https://twitter.com/}{@ankuepfer}}
			\end{column}
		\end{columns}
	}
	{Bach \& K\"{u}pfer}
}
\date{}
\institute{Social Science Data Lab\\ MZES, University of Mannheim \\ February 15, 2023}
\subject{}



\begin{document}

\begin{frame}
  \titlepage
\end{frame}

\nologo{ % no logos except on the title page
	\section{Why Python?}
	\begin{frame}{Outline}
		\tableofcontents[currentsection]
	\end{frame}

	\begin{frame}{Why Python?}
  \begin{columns}
\begin{column}{0.5\textwidth}
	\cemph{Python and R can do the same things, e.g., ...}
 \begin{itemize}
                \item Analyze data using regression and machine learning techniques
                \item Collect data from the web, e.g., through scraping and APIs
                \item Visualize data
            \end{itemize}
\end{column}
\begin{column}{0.5\textwidth}  %%<--- here
    \begin{center}
      \includegraphics[scale=.3]{Day 1/Slides/LaTeX files/why-not-both.jpg} \\
            \tiny{Created with the Imgflip \href{https://imgflip.com/memegenerator}{Meme Generator}}
     \end{center}
\end{column}
\end{columns}
  \end{frame}

	\begin{frame}{Why Python?}
	\small
	\cemph{Python, however, is better suited when ...}
            \begin{itemize}
                \item Working with computer scientists
                \item Using state-of-the-art machine learning, deep learning, natural language processing
                \item Preparing for a data science job outside of academia
                \item General purpose programming
            \end{itemize}
    \small
	Full disclosure: If your typical workflows mainly consist of data prep and analysis using statistical methods, such as all types of regression approaches, including Bayesian statistics, and some ML and NLP, you'll probably be better-off with R.
  \end{frame}


\section{Practical Guide}
	\begin{frame}{Outline}
		\tableofcontents[currentsection]
	\end{frame}

	\begin{frame}{Python}
	\small
	\cemph{Python ...}
            \begin{itemize}
                \item General purpose programming language
                \item Three major versions, only one (Python 3) still maintained
                \item Starting today, obvious choice is Python 3
            \end{itemize}
   		\begin{block}{Is python already installed?}
			\$ python --version
		\end{block}       
  \end{frame}

	\begin{frame}{Versioning of python and packages}
	\small
            \begin{itemize}
                \item Installing the most recent stable release of python and packages is a good starting point
                \item Sometimes, dependencies require other versions of Python and/or packages, however
                \item Solution: Set up different \cemph{virtual environments} 
                \begin{itemize}
                    \item Easy to keep different versions of Python and packages
                    \item Avoid problems with different dependencies and updates
                    \item Can easily switch between virtual environments
                    \item Makes sure that your code will work on collaborators' machines
                \end{itemize}
            \end{itemize}     
  \end{frame}

 	\begin{frame}{Creating and maintaining virtual environments}
	\small
 \textbf{With \gemph{PIP} and \gemph{venv}}
\begin{itemize}
    \item \textbf{venv}: Should come with your python installation
    \item Tool to set up and manage virtual environments
\end{itemize}
\vskip .5cm
                \begin{block}{Create a new virtual environment using \gemph{venv}}
                   \$ python -m venv <directory>
		          \end{block}
                
                \begin{block}{Activate virtual environment}
                   \$ source <directory>/bin/activate
		          \end{block}
                \begin{block}{Deactivate virtual environment}
                   (<directory>) \$ deactivate
		          \end{block}
            \


  \end{frame}

  	\begin{frame}{Versioning of python and packages}
	\small
\textbf{With \gemph{PIP} and \gemph{venv}}
                \begin{itemize}
                    \item \textbf{pip}: \textbf{P}ip \textbf{I}nstalls \textbf{P}ackages
                    \item Standard package manager for python packages
                    \item Should come with your python installation
                    \item Install packages within a virtual environment
                \end{itemize}
                \begin{block}{Install \gemph{pip}}
                   \$ python -m ensurepip --upgrade
		          \end{block} 
                \begin{block}{Install packages using \gemph{pip}}
                   \$ pip install <package name>
		          \end{block} 
  \end{frame}

  	\begin{frame}{Versioning of python and packages}
	\small
                \begin{itemize}
                    \item Sometimes, a \texttt{requirements.txt} file is provided
                    \item Contains required packages and versions
                \end{itemize}
                \begin{block}{Install packages from \gemph{requirements.txt}}
                   \$ pip install -r requirements.txt
		          \end{block} 
            \begin{itemize}
                \item Note
                \begin{itemize}
                    \item Virtual environments are disposable folder structures
                \item Do not put code or data into your virtual environment (folder) manually
                \end{itemize}
            \end{itemize}
  \end{frame}





       	\begin{frame}{Creating and maintaining virtual environments}
\small          
 With \gemph{conda}
\begin{itemize}
    \item "Package, dependency and environment management for any language"
    \item Installation
    \begin{itemize}
        \item Via \cemph{Anaconda}: Python (and R) distribution, popular in data science
        \item Via \cemph{miniconda}: Fewer packages than Anaconda
        \item Install through \href{https://www.anaconda.com}{Anaconda} website
    \end{itemize}
    \begin{block}{Check if conda has been installed}
                   \$ conda --version
		          \end{block} 
\end{itemize}
  \end{frame}

       	\begin{frame}{Creating and maintaining virtual environments}
\small          
 \gemph{Conda}
\begin{itemize}
    \item "Package, dependency and environment management for any language"
    \item Installation
    \begin{itemize}
        \item Via \cemph{Anaconda}: python (and R) distribution, popular in data science
        \item Via \cemph{miniconda}: Fewer packages than Anaconda
        \item Install through \href{https://www.anaconda.com}{Anaconda} website
    \end{itemize}
    \begin{block}{Check if conda has been installed}
                   \$ conda --version
		          \end{block} 
\end{itemize}


  \end{frame}
 
	\begin{frame}{MZES Xaringan Metropolis vs MZES Beamer Metropolis}
	\small
	\cemph{Similarities}
		\begin{itemize}
			\item Both themes use MZES corporate colors.
			\item Both feature title slides with full contact and social media information for multiple authors.
			\item Both embed the MZES Logo on the first slide.
			\item Both use the 16:9 aspect ratio (instead of 4:3), the default on most projectors and screens nowadays.
			\item Both allow for emphases in \cemph{MZES cyan} and \gemph{MZES gold}.
		\end{itemize}
	\end{frame}

		\begin{frame}{MZES Xaringan Metropolis vs MZES Beamer Metropolis}
	\small
	\cemph{Differences}
		\begin{itemize}
			\item Beamer produces static PDF slides, Xaringan produces dynamic, interactive HTML5 slides.
			\item Beamer slides are written in LaTeX, Xaringan slides are written in RMarkdown.
			\item Unlike Beamer, Xaringan allows for R code to be evaluated while knitting the code and for (interactive) R output to be directly embedded in the presentation.
			\item Unlike the Beamer Theme, the Xaringan Theme does not support a permanent footer.
			\item Unlike the Beamer Theme, the Xaringan Theme does not support progress bars on section title slides. Progress bars are only displayed underneath the slide title.
			\item The Xaringan Theme does not support the Beamer Theme's custom blocks, though equivalent classes can be defined in \texttt{mtheme.css}.
			\item The Xaringan Theme does not support the Beamer Theme's optional custom citation commands.
			\item Whereas Beamer directly supports BibTeX references, Xaringan requires a little \href{https://github.com/yihui/xaringan/wiki/Bibliography-and-citations}{work-around} using the R package \href{https://cran.r-project.org/web/packages/RefManageR/index.html}{\texttt{RefManageR}}.
		\end{itemize}
	\end{frame}
	
	\section{Cool Features}
	\begin{frame}{A Sample Slide}
		A sample list:
		\begin{itemize}
			\item \textcolor{mzesgold}{emphasis 1}
			\item \textcolor{mzescyan}{emphasis 2}
			\item \textcolor{mzesdarkgold}{emphasis 3}
		\end{itemize}
	\end{frame}
	
	\begin{frame}{Some Sample Blocks}
		\begin{block}{Remark}
			Sample text
		\end{block}
		
		\begin{alertblock}{Important theorem}
			Sample text
		\end{alertblock}
	\end{frame}
	
%	\begin{frame}{Some Custom Referencing Styles}
%	Lorem ipsum dolor sit amet, consectetur adipiscing elit, sed do eiusmod tempor incididunt ut labore et dolore magna aliqua.\cpsh{Downs1957c} 
	
%	Ut enim ad minim veniam, quis nostrud exercitation ullamco laboris nisi ut aliquip ex ea commodo consequat.\cpsl{Stokes1963} 
	
%	Duis aute irure dolor in reprehenderit in voluptate velit esse cillum dolore eu fugiat nulla pariatur.\cpshl[e.g. ]{KKV, Greene2003, Greene2012}{Wooldridge2002, Gelman2007} 
	
%	\end{frame}
	
	\section{Bibliography}
	\begin{frame}[allowframebreaks]{Bibliography}
	\printbibliography[heading=none]
	\end{frame}
}
\end{document}